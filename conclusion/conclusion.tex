
\begin{savequote}
If the fool would persist in his folly he would become wise.
\qauthor{William Blake}
\end{savequote}

\chapter[Conclusion and Future Directions]{Conclusion and Future Directions}
\label{ChapterConclusion}

Intentional \acf{NVC} messages are an important component of human interaction and a skill which almost all people innately possess. Automatic recognition of \ac{NVC} can enable novel computer interfaces, help computers facilitate human communication and provide a tool for understanding human behaviour. Understanding human behaviour is complicated because both the expression and perception of \ac{NVC} meaning is context specific. Different cultural and social situations have different rules of behaviour. For a realistic automatic system to understand social situations, the effect of context must be considered.

%Summarising thesis findings

The main contributions of this thesis are:

\begin{enumerate}
 \item The recording of a new corpus to study human behaviour in informal conversation. Minimal constraints were imposed on participants to allow spontaneous, natural conversation. The corpus was made publicly available and has already been used by another research group. 
 \item The corpus was annotated by observers based in three different cultures. The paid annotators used an Internet based questionnaire, and their quantised dimensional responses were filtered to ensure high quality labels were produced. The labels used for annotation were based on \ac{NVC} meaning. The annotation data was filtered to produce a consensus set of four component, \continuous of labels which encode \ac{NVC} meaning based on each culture's perception of it. Various types of \ac{NVC} are shown to co-occur, and others to be mutually exclusive. 
 \item A study of automatic \ac{NVC} meaning recognition. The automatic predictions were \continuous labels. Various feature extraction methods and classifiers were compared. Feature selection was employed to improve performance. The temporal envelope of feature components was encoded using simple statistical measures. Culturally specific models of \ac{NVC} perception were shown to be beneficial to recognition performance. Specific areas of the face were identified by the feature selection process that indicate the local areas of the face used in the automatic recognition system for specific \ac{NVC} categories.
 \item Coupling in human behaviour was confirmed and quantified, along with the identifying the areas of the face most relevant for \ac{NVC} recognition. Analysis was performed both quantitatively and by using qualitative visualisation of the facial regions involved. The strongest coupling of behaviour was found to be associated with the mouth and possibly head pose. The use of non-corresponding features resulted in a higher level of correlation for some subjects. Classification using back channel features was shown to be above chance level, although this requires additional experiments to confirm if this finding is statistically significant.
\end{enumerate}

%Discussing the findings more broadly than in earlier chapters
%Coming to conclusions (more generalised than findings)

Historically, many studies of \ac{NVC} and emotion recognition have used posed data sets. However, posed human behaviour is quite different from spontaneous, natural behaviour. The corpus used in this thesis was based in a spontaneous, informal conversational setting. This context is a common social situation and it occurs in every culture. Because \ac{NVC} varies depending on social and cultural rules, recording a corpus in a specific, common situation provides a starting point for the collection of \ac{NVC} data in a variety of other natural situations. Perception of \ac{NVC} also depends on culture, so annotation data was collected from three different cultural groups. Differences in annotations suggests that an automatic system would need to specialise in a particular cultural view point to accurately model and recognise \ac{NVC} in the manner that was comparable to a human. Annotation data was collected from paid volunteers using the Internet, a process sometimes called ``crowd sourcing''. This data collection method has quality problems if used directly and existing filtering methods provided by these services are only relevant for discrete, objective labels. A robust subjective consensus was created using filtering of annotators into trusted and untrusted groups.

Annotation was performed using labels that correspond to meaning in \ac{NVC}. Only a few recent existing studies have attempted to recognise intentional and meaningful communication acts; the handful that do perform automatic recognition of agreement. This thesis broadens the types of \ac{NVC} signals that are recognized to include asking a question, thinking and understanding \ac{NVC}. The \ac{NVC} of thinking, which was previously known to involve gaze aversion, could arguably be considered a mental state and not only a form of \ac{NVC}. However, this gaze pattern may intentionally regulate conversational turn taking and therefore is considered a form of \ac{NVC}. The annotators rated corpus samples for 4 \ac{NVC} categories using quantised, dimensional labels. Most existing facial behaviour recognition studies use discrete, multi class labels. These labels ignore the fact that \ac{NVC} signals and emotions occur in different intensities and these differences may be significant, depending on the intended application. Using dimensional labels retains information of the intensity of the \ac{NVC} signal, which provides a much richer set of labels to use in training an automatic system. 

The quantised, dimensional labels were combined to form consensus labels that were \continuous data. An automatic system was created and trained on the consensus annotation data provided by annotators in three distinct cultures. This enables the automatic system to better model the \ac{NVC} perception of that culture. All other studies have used annotators in a single cultural group or otherwise treated them as culturally homogeneous.
%which is acceptable for labelling which does not have \culturallySpecific characteristics. 
There are a few papers that use distinct groups of annotators, but this has so far been used to partition annotators into naive and expert skilled annotator sets, and in which the labels are objective observations. This thesis considers annotator groups as equally valid observers of subjective behaviour meaning. Only visual data was used in training and testing the automatic system, which forces the system to use visual only \ac{NVC} signals for recognition.

The automatic system evaluation compared several different approaches to determine which approach yielded the best performance. The findings were broadly consistent with existing facial analysis studies: specifically that face shape information was effective for automatic \ac{NVC} recognition. The optimal features were based on distances between pairs of trackers. Combining these features with feature selection, which determines which specific pairs of trackers were important, resulted in a significant performance improvement. Some forms of \ac{NVC} appear to be easier to automatically recognise than others. Even for questioning \ac{NVC}, which was expected to be entirely based on audible signals, the recognition performance was relatively high. This indicates the \ac{NVC} for questioning has a visual component.

Evidence of interpersonal coordination of facial behaviour was found, quantified and visualised. Movement around the mouth was correlated in people that were engaged in conversation. This is likely to be due to mutual smiling. Another coupled behaviour appears to be related to head pitch and may be related to coupled body pose or nodding of the head. 

The features used by the automatic recognition system were identified by feature selection and visualised. This enables an intuitive understanding of the areas of face that were used. Previous approaches have used visualisations that are hard to interpret, such as visualising the single most significant feature. The visualisation method described was based on the overall contribution of all features that are relevant for \ac{NVC} recognition. 
%This results in a better overall appreciation of what areas of the face are significant. 
The resulting visualisations either confirm our expectations, e.g. gaze is important for thinking \ac{NVC}, or suggest new insights into human behaviour, such as the brow region being significant when a question is being asked.

%Giving implications of findings

%Making recommendations

\section{Current Limitations and Future Work}

This thesis provides a first step in automatic understanding of \ac{NVC} signals in a common social setting. However, these findings do not yet have a direct application. This is partly because they enable an entirely new type of computer interface.
% which is suitable for tasks that are currently not possible with existing input devices. 
An interface that is \ac{NVC} aware may be used in conjunction with natural language, such as interacting with computer characters using both verbal and \ac{NVC} signals. Experimental systems for this already exist, such as the SEMAINE \acf{SAL} system \cite{Schroder2011}, which uses manually defined \ac{NVC} rules to predict the mental state of a user and to generate appropriate back channel signals. \ac{SAL} does not attempt to explicitly understand any verbal or \ac{NVC} meaning in the conversation. Applying the work in this thesis would allow meaning to be explicitly understood, but it is possible to sidestep this issue and generate a computer character's response without recognizing \ac{NVC} meaning. %However, this would limit the system's capabilities for complex interactions. 
The work presented in this thesis also shows that \featureGeneration and selection can provide insight into human behaviour. This includes coupling of behaviour as well as identifying which facial areas are involved in \ac{NVC}. Automatic systems provided evidence that relevant information is present in a specific area of the face. However, features used for recognition by automatic systems may be different from the process used by humans to recognise behaviour. 
This area is relatively unexplored, with computer tools only recently being applied to assist humans by automatically finding patterns in human behaviour.

There are several limitations to the work presented in this thesis. The corpus was recorded in a single social and cultural context. Having culturally distinct sets of \ac{NVC} expression would allow cultural differences to be automatically analysed. This is related to Hassan's work on cross language emotion recognition based on audio \cite{Hassan2012}. Additional cultural data would enable automatic systems to be culturally specialised and therefore to improve performance. However, there are so many different cultural and social situations, that this process cannot be performed exhaustively. There are likely to be similarities within groups of social situations and this would likely reduce the scale of the problem of contextual differences. The different levels of expressivity of the conversation participants was not considered. Normalising or removing these variations, based on annotation label data, could improve person independent recognition. However, the system would then require personal adaptation to the subject's style of expression. This improvement by familiarity of personal style is commonly used by humans but is currently not considered by automatic systems. The corpus has been recorded in a laboratory environment but may modify human behaviour in comparison to behaviour in the field. A better data set might be obtained by recording in a domestic or work environment without the experimenters attempting to modify the conditions in which \ac{NVC} is expressed. The latest datasets have begun to adopt this approach. However, recording \ac{NVC} without the knowledge of participants in an uncontrolled environment has serious ethical concerns, particularly if the corpus is shared and used by multiple research groups for the purposes of performance benchmarking. It may be possible to use geo-centric (wearable) cameras to record human behaviour in the field but the resulting data would be challenging to process due to lighting changes, pose, the changing environment and ethical data protection issues.

Cultural differences exist in the annotation data. However, it is difficult to be sure if this is caused by perceptional differences or other factors. There are cultural differences in language usage and computer skills, as well as different samples of the population participating in annotation; these factors might also be causing changes in annotation perception. Some of these issues could be addressed by gathering demographic and personality data. Also, annotators could be grouped using unsupervised clustering and a specialised model trained in the annotator set. %, but this would require a additional annotation data to form stable or meaningful groups of annotators.
Further work could be done to validate the questionnaire, based on objective assessments and ensuring this is consistent across cultures.

Only four \ac{NVC} signals were annotated and automatically recognized. However, there is no clear upper limit for the number of types of \ac{NVC} that may be expressed. \ac{NVC} can be hard to notice that it is even occurring, such as conversation turn taking where the meaning is rarely explicitly expressed. Many other \ac{NVC} signals could be proposed but \ac{NVC} signals are not expressed in isolation. Dimensionality reduction may enable a relatively comprehensive coverage with a finite number of \ac{NVC} labels.

Many \ac{NVC} signals evolve in time and cannot be accurately recognized from a photograph or single frame of video. The system in this thesis encodes temporal variation by simple statistical measures for manually selected video clips. However, it is not clear how to apply this method to a real time system in which there is no clear start or end of \ac{NVC} expression. A technique to select an optimal temporal window size, or another way to encode temporal information needs to be applied to create an effective real time system. The temporal encoding used in this thesis is simple. %and this is partly because of the high variability of human behaviour. 
No other consistent \ac{NVC} behaviours were found that required a more sophisticated encoding method, but it is quite possible these exist. Further work into identifying and temporally modelling the behaviour over several seconds (or longer) would be useful for many practical applications.

Only visual features are analysed, but \ac{NVC} is closely associated with both the non-verbal aspect of voice, as well as the verbal meaning. Also, only facial features are investigated but future work might employ body pose, shoulders, hand shapes or other visual features as part of \ac{NVC}. The role of clothing, proximity and other non facial \ac{NVC} signals might also be useful, if automatically recognized. Future work may consider the close relationship between speech recognition and \ac{NVC} recognition. Work has already been done on audio-visual speech recognition but little has been done on multi-modal \ac{NVC} recognition. Combining \ac{NVC} with verbal meaning may result in an increase in verbal recognition accuracy due to the close association between words, emotions and \ac{NVC}.

Tracking is performed using a method that requires annotation of multiple frames and occasionally needs manual re-initialisation after occlusions. This is not suitable for a production system that operates on previously unseen people. A system that uses facial detection to reinitialise would be more robust to occlusions. Although the tracker is relatively robust to limited pose change, large pose change still causes significant tracker drift due to the radical change in appearance. Large pose changes would need to be tolerated by a system that needs to operate in situations in which users move around freely.

%Suggesting areas of future research.

This thesis has used various methods to recognise \ac{NVC} in natural data. Many other techniques exist, particularly for emotion recognition. However, most existing techniques have only been applied to posed data. It would be informative to see which techniques generalise from posed data to natural situations. This is a necessary step if a technique is to be applied to applications in an unconstrained environment. In particular, there are many \featureGeneration methods. It is still unclear if shape or appearance is more useful for automatic facial analysis in general but previous studies have concluded that fusion of multiple strategies is likely to provide optimal performance. The current geometric features are distance based and are sensitive to scale changes. These features may be generalised to include other types of geometric arrangements, as well as made robust to scale and pose change.

%Future experiments based on manipulated video? \cite{Boker2009} Is this a natural context?

%Some interesting future areas are in \cite{GaticaPerez2009}

%Future directions in SSP \cite{Vinciarelli2008}

%The value of using an intermediate representation? How similar are geometric features to FACS? FACs is normally used as binary classes, which might not be as useful as continuous data.

The annotation was performed by observers using Internet based tools. The introduction of this technique has already had an impact in the availability of training data in many fields, such as object recognition, and it is beginning to be applied to human behaviour. However quality issues are still a problem as some workers do not cooperate with the task. This issue is difficult to address for subjective questions or if the responses are based on \continuous scales. Given recent advances in crowd sourcing services, it should be possible to target questionnaires to particular regions and perhaps to gather demographic and personality based information. This will provide a much richer understanding of perception differences based on context.

Research based on a practical application of \ac{NVC} recognition may provide a set of constraints for the problem, as well as providing clues as to the most suitable environment for recording and which labels are appropriate for annotation. The XBox Kinect has provided an application for body pose estimation, and with it a set of assumptions for the hardware platform, distance from the camera and domestic operating environment. In a similar way, a concrete application for \ac{NVC} may provide guidance as to suitable constraints and assumptions that can be used, as well as the factors that require system robustness.

%\section{Last Word}

Automatic \ac{NVC} recognition remains a relatively new and open area of enquiry. It is also fragmented between various academic disciplines, each with their own practices, jargon and methodologies. The field is likely to benefit from exchange of ideas from these different groups, including between computer vision, psychology, anthropology, linguistics and engineering. However, the goal of socially aware computing in every-day life is still some way away.
